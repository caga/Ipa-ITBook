%\documentclass[11pt,fleqn]{book}
%\usepackage{standalone}
%\usepackage{subcaption}
%\usepackage{rotating}
%\usepackage{verbatim}
%\input{../../structure} % Insert the commands.tex file which contains the majority of the structure behind the template

%\begin{document}

\chapter{TMUX}

\section{Temel İşlemler}\index{Temel İşlemler}

Eğer geliştirme işiniz için linux kullanıyorsanız, Tmux a ihtiyacınız var demektir. Çünkü işletim sistemi komut penceresinde yaşar ve nefes alır ve komut penceresini kullanmanın en esnek yollarından biri Tmux dur.
\subsection {Konfigurasyon}
Tmux da konfigurasyon ev dizinindeki  \textbf{\emph{.tmux.conf}} gizli dosya üzerinden yapılır. Ana kumanda tuşunun ctrl-a ya çevrildiği, vim tuş düzenin ayarlandığı, mouse kullanımının açıldığı ve kopyalama kısa yollarının değiştirildiği bir configurasyon örneği aşağıda verilmiştir.
\lstinputlisting[breaklines]{/home/cagatay/.tmux.conf}
%\end{document}
